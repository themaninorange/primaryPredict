\documentclass[xcolor=dvipsnames]{beamer}

\usepackage[utf8]{inputenc}
\usepackage{booktabs, subfig, graphicx, multirow}
\usepackage{tikz}
\usepackage[absolute,overlay]{textpos} 
\usepackage{soul}
\usetheme{PaloAlto}
\definecolor{TarletonPurple}{RGB}{79, 45, 127}
\usecolortheme[named=TarletonPurple]{structure}
\newenvironment{reference}[2]{
\begin{textblock*}{\textwidth}(#1,#2)              
  \footnotesize\it\bgroup\color{red!50!black}}{\egroup\end{textblock*}}


 \graphicspath{{./Images/}}
%\setbeamercolor{frametitle}{bg=RoyalPurple}
%\setbeamercolor{sidebar}{bg=RoyalPurple}
%\setbeamercolor{logo}{bg=RoyalPurple!70!black!}



%% TITLE PAGE INFO%%
\title{If Twitter Could Vote}
\author{Joseph Brown, Mikaela Jordan, Adam Swayze}
\institute{Tarleton State University}
\date{January 7, 2017}

\begin{document}
\makeatletter
\def\beamer@framenotesbegin{
\begin{reference}{1mm}{1mm}
\tikz\node[opacity=0.8]{\includegraphics[scale=0.25]{Tarleton_State_University}};
\end{reference} 

}
\makeatother
\frame{\titlepage}



%\begin{frame}
%\frametitle{Overview \hfill \includegraphics[scale=.04]{map.png}}
%
%\begin{itemize}
%\item Predicting Primary Results
%\begin{itemize}
%\item Use Twitter data to create a model that predicts primary results for each county.
%\end{itemize}
%\end{itemize}
%\end{frame}

\begin{frame}
\frametitle{Predicting Primary Results \hfill \includegraphics[scale=.15]{fivethirtyeight-logo}}

\begin{itemize}
\item In Primaries, polls change fast and are inaccurate	
\end{itemize}


\begin{figure}
\includegraphics[scale=.35]{poll2.png}
\caption{Late March/Early April Polls for the Wisconsin Republican Primary from 538}
\end{figure}


\end{frame}

\begin{frame}
\frametitle{Predicting Primary Results \hfill \includegraphics[scale=.15]{fivethirtyeight-logo}}

\begin{itemize}
\item In \st{Primaries}, polls change fast and are inaccurate	
\end{itemize}


\begin{figure}
\includegraphics[scale=.35]{poll2.png}
\caption{Late March/Early April Polls for the Wisconsin Republican Primary from 538}
\end{figure}


\end{frame}

\begin{frame}
\frametitle{Alternative Primary Predictions \hfill \includegraphics[scale=.035]{facebook.png}}
\section{Polling and Predictions}
\begin{itemize}
\item Counties colored by Candidate with most Facebook likes 
\end{itemize}
\begin{center}

\includegraphics[scale=.14]{allmap.png}
\end{center}
\end{frame}
%
%\begin{frame}
%\frametitle{Candidates and Social Media  \hfill \includegraphics[scale=.015]{likes.png}}
%
%\begin{itemize}
%\item The Final Five
%\end{itemize}
%\begin{center}
%
%\includegraphics[scale=.14]{currentcandidate.png}
%\end{center}
%\end{frame}

\begin{frame}
\frametitle{Candidates and Social Media  \hfill \includegraphics[scale=.015]{likes.png}}
\begin{itemize}
\item Democratic Candidates
\end{itemize}
\begin{center}

\includegraphics[scale=.14]{demmap.png}
\end{center}
\end{frame}

\begin{frame}
\frametitle{Candidates and Social Media  \hfill  \includegraphics[scale=.015]{likes.png}}

\begin{itemize}
\item Republican Candidates 
\end{itemize}
\begin{center}

\includegraphics[scale=.14]{repmap.png}
\end{center}
\end{frame}


\begin{frame}
\frametitle{Natural Language Processing}
	\begin{itemize}
		\item Basic Idea
		\begin{itemize}
			\item Computers understanding language
		\end{itemize}
		\pause
		\item Several Tasks in Natural Language Processing
		\begin{itemize}
			\item Question Answering
			\item Automatic Summarization
			\item Sentiment Analysis
		\end{itemize}
	\end{itemize}
\end{frame}



\begin{frame}
\frametitle{Popular Word Usage   \hfill \includegraphics[scale=.1]{twitterlogo.png}}
\begin{figure}[h!]
\subfloat[][]{\includegraphics[scale=.38]{newwordclouddem}\label{Keyword ``Democrat}}
\hspace{1mm}
\subfloat[][]{\includegraphics[scale = .4]{newwordcloudrep}\label{Keyword ``Republican}}
\caption{Word Clouds for Twitter Searches with Keywords ``Democrat'' and ``Republican''}
\end{figure}
\end{frame}

%\begin{frame}
%\frametitle{Popular Word Usage   \hfill \includegraphics[scale=.01]{DemocraticLogo.jpg}}
%\includegraphics[scale=.55]{wcrepwithRT.png}	
%\end{frame}

\begin{frame}
\frametitle{Design of the Primary Prediction \\ \small In a Perfect World}
\begin{itemize}
	\item Created a dataset with the names of all counties in US, their respective FIPS codes, and the recorded majority winners for each county 
	\begin{itemize}
	\item Unique identification number for each county in US
	\end{itemize}
	\pause 
	\item Found a dataset with area and centroid of each county in US
%	\begin{itemize}
%		\item Create a circle around centroid with radius equal to the $\sqrt{\text{Area}_{\text{county}}}$
%	\end{itemize}
	\pause
	\item Search for tweets with specified keywords in every county of US
\begin{center}
\resizebox{.9\linewidth}{!}{
\begin{tabular}{|c|c|c|c|c|c|}
	\hline
	Candidates & Bernie Sanders & Hillary Clinton & Donald Trump & John Kasich & Ted Cruz \\
	\hline 
	Keywords & ``Bernie"  & ``I'm with Her" & ``Trump" & ``Kasich" & ``Cruz" \\
	&``Sanders" & "HillaryClinton" & ``Donald" & ``JohnKasich" & ``TedCruz \\
	& ``Feel The Bern'' & ``Hillary2016'' & ``Make America Great Again'' & ``Kasich2016'' & ``Trust Ted'' \\
	& ``Bernie2016'' & ``Clinton'' & ``Trump2016'' && ``TrustTed'' \\
	& & & ``DonaldTrump'' & & ``Cruz2016'' \\
	\hline
\end{tabular}}
	
\end{center} \pause
	\item Merge results data frame with tweets data frame by FIPS code
	\end{itemize}
\end{frame}



%\begin{frame}
%\frametitle{Design of the Primary Prediction \\ \small In a Perfect World}
%\begin{itemize}
%\item Merge data frame with results with data frame of tweets by FIPS code
%	\begin{itemize}
%	\item Unique identification number for each county in US
%	\end{itemize}
%	\end{itemize}
%\end{frame}

\begin{frame}
\frametitle{Design of the Primary Prediction \\ \small In a Perfect World}
\large{An Example}\\
 An original tweet from our collection: 
\begin{center}
\includegraphics[scale=0.5]{uncleantweets3.png}
\end{center} \pause 
	\begin{itemize}
	\item Cleaning Tweets
	\begin{itemize}
		\item Remove punctuation \pause
		\item Remove emojis \pause 
		\item Remove Stopwords \pause 
		\item Make everything lower case \pause 
	\end{itemize}
	\end{itemize}
\begin{center}
\includegraphics[scale=0.5]{CleanTweet.png}
\end{center}
\end{frame}

\begin{frame}
\frametitle{Design of the Primary Prediction \\ \small In a Perfect World}
\begin{itemize}
	\item Create a Document-Term Matrix
	\begin{table}[h!]
	\resizebox{\linewidth}{!}{
	\begin{tabular}{|c|c|c|c|c|}
	\hline
	Doc Name & Term 1 & Term 2 & \dots & Term $n$ \\
	\hline
	Doc 1 & Freq(T1 in D1) & Freq(T2 in D1) & \dots & Freq(T$n$ in D1) \\
	\hline
	\vdots & & & & \\
	\hline
	Doc $m$ & Freq(T1 in D$m$) & Freq(T2 in D$m$) & \dots & Freq(T$n$ in D$m$) \\
	\hline
	\end{tabular}}
	\end{table} \pause 
	
	\begin{table}[h!]
	\resizebox{0.85\linewidth}{!}{
	\begin{tabular}{|c|c|c|c|c|}
    \hline 
	\small Doc Name & \small Term 1 (Trump) & \small Term 2  & \dots & \small Term $n$ \\
	\hline
	\small My Tweet & 5 & 0 & \dots & 0 \\
	\hline
	\end{tabular}}
	\end{table}

	\item Attach labels to each tweet based on location of tweet \pause 
	\item Use machine learning algorithms to predict majority winners of each tweet's county
\end{itemize}
\end{frame}

\begin{frame}
\frametitle{Results \hfill \includegraphics[scale=.12]{democraticlogo2.png}}
\section{Results}
\subsection{Democrats}
\begin{table}
\centering
\resizebox{.4\linewidth}{!}{
\begin{tabular}{cc|c|c|}
\cline{3-4}
 & & \multicolumn{2}{|c|}{Predicted} \\
 \cline{3-4}
 & & Bernie & Hillary \\
 \hline
 \multicolumn{1}{|c|}{\multirow{2}{*}{Actual}} & \multicolumn{1}{|c|}{Bernie} & 431 & 3332 \\
 \cline{2-4}
 \multicolumn{1}{|c|}{} & \multicolumn{1}{|c|}{Hillary} & 148 & 10212 \\
 \hline
\end{tabular}}
\caption{Support Vector Machine Confusion Matrix}
%This model has an accuracy of 75.35934\%
\end{table}
\begin{table}
\centering
\resizebox{.4\linewidth}{!}{
\begin{tabular}{cc|c|c|}
\cline{3-4}
 & & \multicolumn{2}{|c|}{Predicted} \\
 \cline{3-4}
 & & Bernie & Hillary \\
 \hline
 \multicolumn{1}{|c|}{\multirow{2}{*}{Actual}} & \multicolumn{1}{|c|}{Bernie} & 434 & 3329 \\
 \cline{2-4}
 \multicolumn{1}{|c|}{} & \multicolumn{1}{|c|}{Hillary} & 132 & 10228 \\
 \hline
\end{tabular}}
\caption{Neural Network Size 6 Confusion Matrix}
%Accuracy rate of 75.49388\%
\end{table}
\begin{table}
\centering
\resizebox{.4\linewidth}{!}{
\begin{tabular}{c|c}
 & Accuracy Rate \\
 \hline
Support Vector Machine & 75.36\% \\
\hline
Neural Network & 75.50\% \\
\end{tabular}}
\caption{Accuracy Rates of Both Models}
\end{table}

\end{frame}

\begin{frame}
\frametitle{Results \hfill \includegraphics[scale=.1]{republicanlogo.png}}
\subsection{Republicans}
\begin{figure}[h!]
\includegraphics[scale = 0.2]{RFmodelaccuracy250.png}
\caption{A plot of classification accuracy versus probability threshold for Random Forest Model}
\end{figure}
\begin{table}
\centering
\resizebox{.4\linewidth}{!}{
\begin{tabular}{cc|c|c|}
\cline{3-4}
 & & \multicolumn{2}{|c|}{Predicted} \\
 \cline{3-4}
 & & Trump & Not Trump \\
 \hline
 \multicolumn{1}{|c|}{\multirow{2}{*}{Actual}} & \multicolumn{1}{|c|}{Trump} & 9517 & 42 \\
 \cline{2-4}
 \multicolumn{1}{|c|}{} & \multicolumn{1}{|c|}{Not Trump} & 1695 & 58 \\
 \hline
\end{tabular}}
\caption{Random Forest with 250 Trees Confusion Matrix.  Has an accuracy of 84.64\%}

\end{table}
\end{frame}

\begin{frame}
\frametitle{Limitations with Social Media \hfill \hfill \includegraphics[scale=.1]{twitterlogo.png}}
\begin{itemize}

	\item According to the Pew Research Center, only 23\% of Americans use Twitter \pause  
	\item Less than 5\% of all tweets are georeferenced \pause 
	\item Can only get tweets up to a week prior of collection time \pause 

	
\end{itemize}

\end{frame}

\begin{frame}
\frametitle{Future Work}
\section{Future Work}
\begin{itemize}
\item Continue to try to improve our models 
\item Compare to traditional polling results
\item Does Twitter have an advantage over traditional polling?
\end{itemize}
\end{frame}

\begin{frame}
\frametitle{References}
\section{References}
\small{We'd like to thank the Office of Student Research and Creative Activity at Tarleton State University}
\begin{center}
\includegraphics[scale=.10]{thank-you}
\end{center}
\tiny
\noindent Benyamin, Dan.``A Gentle Introduction to Random Forests, Ensembles, and Performance Metrics in a \\
\hspace{1cm}Commercial System - Blog \& Press''  \textit{Citizen and Net.} N.p., 09 Nov. 2012. \\
\hspace{1cm}Web. 28 Mar. 2016. \\
Phillips, Winfred.  ``Introduction to Natural Language Processing.'' \textit{Consortium on Cognitive Science} \\
\hspace{1cm}\textit{Instruction.} The MIND Project, 2006.  Web.  16 Mar 2016. \\
Raine, Lee. ``Social Media and Voting." \textit{Pew Research Center Internet} \textit{Science Tech RSS.} Pew Charitable \\
\hspace{1cm}Trusts, 05 Nov. 2012. Web. 28 Mar. 2016.
Weston, Jason. ``Support Vector Machines (and Statistical Learning Theory)." ComputerScience4701. \\
\hspace{1cm}Columbia University, New York. \textit{Columbia University}. Web. 28 Mar. 2016.\\
``FiveThirtyEight.'' \textit{FiveThirtyEight.} Nate Silver, n.d. Web.  28 Mar 2016. \\
``2016 Primary Election Results: President Live Map by State, Real-Time Voting Updates'' \textit{Election Hub.} \\ 
\hspace{1cm}Politico LLC, 27 Mar. 2016.  Web.  28 Mar. 2016. \\
\end{frame}





\end{document}